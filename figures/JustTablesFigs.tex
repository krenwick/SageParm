\documentclass[]{article}
\usepackage[T1]{fontenc}
\usepackage[]{dcolumn}
\usepackage[]{booktabs}
\usepackage[labelsep=period]{caption}
\usepackage{graphicx} 
\usepackage{tabularx}
\usepackage{threeparttable} % for footnotes in table
\usepackage[letterpaper, portrait, margin=.75in]{geometry}
\usepackage{gensymb}
\usepackage{textcomp}
\usepackage[section]{placeins}
\usepackage[round]{natbib}
%\addbibresource{SageParm.bib}
%\captionsetup[table]

%opening
\title{}
\author{}

\begin{document}
	
\section{Tables}

\begin{table}[ht]
	\begin{threeparttable}
	\caption{Standard values for the parameters evaluated in the sensitivity analysis, along with the minimum and maximum values tested and reference for that range.} 
	\begin{tabular}{lcccl}
		\toprule
		Parameter & Standard & Min & Max & Reference \\ 
		\midrule
		Specific leaf area (m\textsuperscript{2} kgC\textsuperscript{-1}) & 30 & 6 & 21 & Full range from literature\tnote1 \\
		Leaf to sapwood area ratio & 6000 & 1350 & 5220 & min/max measured\tnote2 \\
		Min. canopy conductance (mm s\textsuperscript{-1}) & 0.5 & 0.35 & 0.65 & standard +/- 30\% \\
		Max. leaf to root ratio & 1 & 0.5 & 1 & full range in global PFTs \\
		Min. growth efficiency (kgC m\textsuperscript{2} leaf\textsuperscript{-1} yr\textsuperscript{-1}) & 0.08 & 0.06 & 0.1 & standard +/- 30\% \\
		Fraction of roots in upper .5m soil & 0.6 & 0.6 & 1 & full range in global PFTs \\
		Sapwood turnover rate (proportion yr\textsuperscript{-1}) & 0.1 & 0.05 & 0.1 & full range in global PFTs \\
		Max. establishment rate (individuals m\textsuperscript{-2} yr\textsuperscript{-1}) & 0.2 & 0.05 & 0.2 & full range in global PFTs \\
		Min. temp. for photosynthesis (\celsius) & -4 & -5.2 & -2.8 & standard +/- 30 \% \\
		Low temp. for optimal photosynthesis (\celsius) & 10 & 7 & 13 & standard +/- 30 \% \\
		High temp. for optimal photosynthesis (\celsius) & 25 & 17.5 & 32.5 & standard +/- 30 \% \\
		Max. temp. for photosynthesis (\celsius) & 38 & 26.6 & 49.4 & standard +/- 30 \% \\
		k\_chillb: coefficient in budburst chilling eq. & 100 & 100 & 600 & full range in global PFTs \\
		GDD\textsubscript{5} sum for full leaf cover & 200 & 100 & 300 & standard +/- 50\% \\
		Leaf longevity (yrs) & 0.5 & 0.25 & 1 & full range of reasonable values \\
	\bottomrule
	\end{tabular}
\begin{tablenotes}
	\item[1] Add citations here
	\item[2] \citet{Ganskopp1986}
\end{tablenotes}
\end{threeparttable}
\end{table}

\begin{table}[ht]
	%\centering
	\begin{threeparttable} % wraping in this makes caption the width of table
	\caption{RPCCs for the six most influential model parameters, ranked by the mean across all output variables. Only parameters where |RPCC| > 0.2 for at least one variable are shown. Each RPCC is the mean across all four sites. Because the sign of RPCC differed among variables, we show the mean of the absolute values of RCPPs.} 
	\begin{tabular}{lrrrrrrrr}
		\toprule
		% & is a stand-in for a column
		&& \multicolumn{2}{c}{Ecosystem} & \multicolumn{4}{c}{Sagebrush} \\
		\cmidrule(lr){3-4} \cmidrule(lr){5-8}
		%Variable  &  \multicolumn{1}{r}{Mean}  &  \multicolumn{1}{r}{Std Dev}  &       \multicolumn{1}{r}{Mean}  &  \multicolumn{1}{r}{Std Dev}  &  \multicolumn{1}{r}{Difference}  \\
		%\midrule
		% latex table generated in R 3.3.2 by xtable 1.8-2 package
% Tue Jun 13 10:07:31 2017
Parameter & mean & NEE & GPP & FPC & LAI & NPP & CMASS \\ 
  \midrule
sla & 0.62 & 0.22 & -0.42 & 0.75 & 0.79 & 0.77 & 0.78 \\ 
  root\_up & 0.48 & -0.06 & -0.19 & -0.34 & -0.37 & -0.39 & -0.38 \\ 
  ltor\_max & 0.33 & 0.03 & -0.18 & 0.40 & 0.46 & 0.45 & 0.45 \\ 
  phengdd5 & 0.21 & -0.08 & 0.11 & -0.23 & -0.26 & -0.29 & -0.28 \\ 
  latosa & 0.20 & 0.01 & -0.07 & -0.35 & -0.17 & -0.21 & -0.33 \\ 
  pstemp\_max & 0.19 & -0.01 & -0.03 & 0.25 & 0.25 & 0.24 & 0.25 \\ 
  pstemp\_hi & 0.16 & 0.02 & 0.06 & -0.21 & -0.22 & -0.20 & -0.21 \\ 
  est\_max & 0.13 & 0.04 & -0.02 & 0.20 & 0.16 & 0.17 & 0.16 \\ 
   \bottomrule

		%\bottomrule
	\end{tabular}
\end{threeparttable}
\end{table}

\begin{table}[ht]
	\begin{threeparttable}
	\caption{RPCCs for the seasonal variables at each site. Seasonal variables represent the proportion of annual GPP that occurs in each season. Only parameters where |RPCC| > 0.2 for at least one site and season are shown.} 
	\begin{tabularx}{\textwidth}{lrrrrrrrrrrrr}
		\toprule
		% & is a stand-in for a column
		& \multicolumn{4}{c}{Spring} & \multicolumn{4}{c}{Summer} & \multicolumn{4}{c}{Fall} \\
		\cmidrule(lr){2-5} \cmidrule(lr){6-9} \cmidrule(lr){10-13}
		%Parameter & wbsec & losec & burn & mbsec & wbsec & losec & burn & mbsec & wbsec & losec & burn & mbsec   \\
		%\midrule
		% latex table generated in R 3.3.2 by xtable 1.8-2 package
% Tue Jul 18 11:55:22 2017
  Parameter & wbsec & losec & burn & mbsec &
                      wbsec & losec & burn & mbsec &
                      wbsec & losec & burn & mbsec \\ \midrule
sla & -0.64 & -0.69 & -0.67 & -0.64 & 0.58 & 0.60 & 0.50 & 0.40 & -0.04 & 0.34 & 0.56 & 0.52 \\ 
  root\_up & -0.36 & 0.74 & 0.66 & 0.67 & 0.22 & -0.65 & -0.57 & -0.56 & 0.17 & -0.43 & -0.53 & -0.47 \\ 
  ltor\_max & -0.21 & -0.26 & -0.23 & -0.21 & 0.15 & 0.15 & 0.13 & 0.06 & 0.07 & 0.17 & 0.22 & 0.21 \\ 
   \bottomrule

		%\bottomrule
	\end{tabularx}
\end{threeparttable}
\end{table}

\begin{table}[ht]
	\begin{threeparttable}
	\caption{RPCCs for the eight model parameters where |RPCC| > 0.2 for annual GPP or LAI. Each RPCC is the mean across all four sites.} 
	\begin{tabular}{lrr}
		\toprule
		% latex table generated in R 3.3.2 by xtable 1.8-2 package
% Mon Aug 28 12:25:34 2017
Parameter & GPP & LAI \\ 
  \midrule
apheng & 0.84 & -0.37 \\ 
  sla & 0.22 & 0.83 \\ 
  phen\_winter & 0.41 & 0.64 \\ 
  root\_up & -0.27 & -0.37 \\ 
  ltor\_max & 0.08 & 0.48 \\ 
  phen5g & -0.41 & 0.03 \\ 
   \bottomrule

	\end{tabular}
\end{threeparttable}
\end{table}

\section{Figures}
\begin{figure}[!htbp]
%\centering\captionsetup{}
\begin{measuredfigure}
	\includegraphics[width=\linewidth]{GPP_LAI_origpheno.pdf}
	\caption{Modeled GPP (a-d) and LAI (e-h) compared to GPP measured at flux sites and LAI derived from MODIS data. Panels represent different sites. The black lines represent the reference data while the colored lines represent model output derived using standard parameters and each of the three phenology types available in LPJ-GUESS.}
	\label{fig:origpheno}
	\end{measuredfigure}
\end{figure}

\section{References}
\bibliographystyle{ecology}
\bibliography{SageParm}

\end{document}
